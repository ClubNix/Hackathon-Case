\documentclass{article}

\usepackage[utf8]{inputenc}
\usepackage[T1]{fontenc}
\usepackage[french]{babel}
\usepackage[hidelinks]{hyperref}
\usepackage{amsmath}
\usepackage{fullpage}
\usepackage{microtype}

\title{Projet Hackathon}
\author{Club*Nix}

\begin{document}

\maketitle

\section{Qu'est-ce qu'un Hackathon?}

\paragraph{}
Un Hackathon consiste à regrouper sur une période donnée des personnes férues
de nouvelles technologies afin de réaliser des projets sur un thème donné. Nous
aimerions créer un Hackathon au sein de l'école sur le thème des objets
connectés du 2 au 3 avril.

\section{Pourquoi faire un hackathon?}

\paragraph{}
Le but est de pousser les élèves à créer des projets innovants et à mettre en
pratique leur compétences techniques dans un cadre de projets concrets et
motivants. Ils sont donc placés dans des conditions plus proches des conditions
réelles de travail.

\paragraph{}
C'est aussi une occasion de découvrir le travail avec des sujets non guidés.
Sans orientations de la part des professeurs, les participants vont ainsi par équipe 
discuter en amont de la manière d'opérer et débattre des meilleures
possibilités.

\section{Qui participe?}

\paragraph{}
Au niveau opérationnel, les organisateurs seront en grande majorité des
membres du Club*Nix, aidés par des professeurs qui nous fournissent leur
soutient.

\paragraph{}
Quant aux participants au Hackathon, ils seront restreints aux élèves de
l'ESIEE Paris, ce projet étant le premier Hackathon organisé.


\section{Demande de ressources}
\begin{description}
	\item[Trois salles info:] de préférence communicantes (possiblement en 5000)
	\item[Une ou deux salles pour les tutos:] projecteur et tableau blanc
	\item[Une salle de repos (canapés, foyer):] car nous aimerions que le Hackathon soit sans interruption du samedi matin au dimanche soir
	\item[Un serveur:] pour fournir des sessions aux participants
	\item[Une connexion filaire:] pour les ordinateurs portable des participants
	\item[Des objets connectés] (lesquels?, dépendant des sujets)
	\item[Du matériel pour les hypothétiques workshops]
	\item[Un serveur pour le site web du Hackathon:] le site sera probablement hébergé sur un serveur du club
	\item[Des boissons, sodas, etc…] (autorisation?)
	\item[De la nourriture:] pizza, sandwichs, probablement fast-food en majorité
	\item[Des personnes pour organiser la logistique du Hackathon sur le moment:] membres du club
	\item[Des personnes pour fournir une aide technique aux participants:] membres du club
	\item[Gardiens]
	\item[Un jury entreprise]
	\item[Un jury enseignant]
\end{description}

\end{document}