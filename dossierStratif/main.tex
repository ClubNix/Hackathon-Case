\documentclass{article}

\usepackage[utf8]{inputenc}
\usepackage[T1]{fontenc}
\usepackage[french]{babel}
\usepackage[hidelinks]{hyperref}
\usepackage{amsmath}
\usepackage{fullpage}
\usepackage{microtype}

\title{Proposition d'organisation d'un Hackathon à ESIEE Paris}
\author{Club*Nix\footnote{contact:
		\texttt{\href{mailto:clubnix@esiee.fr}{clubnix@esiee.fr}},
		\texttt{\href{mailto:remi.nicole@edu.esiee.fr}{remi.nicole@edu.esiee.fr}}}\\
	avec le soutien du département informatique (Jean Cousty)}

\begin{document}

\maketitle

\section{Informations importantes}

\begin{itemize}
	\item Un ``concours'' de projets informatiques à ESIEE Paris
	\item Pour les étudiants d'ESIEE Paris
	\item Sur le thème des objets connectés et du data science
	\item Réalisés en 30H du 2 au 3 avril 2016
	\item 50 participants
\end{itemize}

\section{Introduction}

\paragraph{}
Un Hackathon consiste à regrouper sur une période donnée des personnes férues
de nouvelles technologies afin de réaliser des projets informatique sur un
thème donné. Nous aimerions créer un Hackathon au sein de l'école sur le thème
des objets connectés.

\paragraph{}
Le but est de pousser les participants à créer des projets innovants et à
mettre en pratique leur compétences techniques dans un cadre de projets
concrets et motivants. Les participants se répartissent en équipe. Ils sont
donc placés dans des conditions plus proches des conditions réelles de travail.

\paragraph{}
C'est aussi une occasion de découvrir le travail avec des sujets non guidés.
Sans orientations de la part des professeurs, les participants vont ainsi par
équipe discuter en amont de la manière d'opérer et débattre des meilleures
possibilités.

\paragraph{}
Un amphi d'information par le Club*Nix à destination des étudiants ESIEE Paris
a été organisé. Nombre de participants: . Cela laisse penser que
l'organisation d'une telle manifestation pourrait être un succès.

% TODO:
Prix décernés par le jury

\section{Planning prévisionnel}
2 avril à 8H du matin au 3 avril 2016 à 20H, sans interruption, y compris
pendant la nuit.

% TODO:
Tableau: planning prévisionnel

\section{Participants et demande d'autorisation d'accès}

\paragraph{Organisateurs.} Au niveau opérationnel, les organisateurs seront des
étudiants d'ESIEE Paris, en grande majorité des membres du Club*Nix. L'objectif
est de constituer une équipe d'environ dix organisateurs. Cette équipe sera
coordonnée par Rémi Nicole. Elle comprend pour l'heure: Élodie Caroy, Romain
Gille, \ldots

\paragraph{Développeurs de projets.} Tous les  élèves d'ESIEE Paris pourront
participer au Hackathon afin de développer un projet à soumettre au jury. En
revanche, la participation n'est pas ouverte à des extérieurs. L'objectif est
de réunir entre quinze et trente développeurs.

\paragraph{Jury.} Un jury sera constitué d'une dizaine de personnes, incluant
des enseignants d'ESIEE Paris et des personnalités extérieures.  Nous
envisageons de contacter les personnes suivantes pour leur demander de faire
partie du jury:

\begin{itemize}
  \item un membre de la direction (Dominique Perrin, Eric Rahingomanana, Didier
	  Degny)
  \item un enseignant en management et sciences humaines (Krys Markowski, Eric
	  Wirth, JM Pointet)
  \item Un ou deux enseignants S\&T d'ESIEE Paris
  \item Un ou deux enseignants extérieurs à ESIEE Paris (EPITA, Université
	  Marne)
  \item 4--5 personnalités du monde de l'entreprise
\end{itemize}

\paragraph{}
\textbf{L'ensemble des organisateurs, développeurs et membres du jury devra
	accéder au site du Hackathon du 2 au 3 avril sans interruption, y compris
	la nuit.}

\paragraph{}
La liste des participants sera fournie au plus tard le XX.

\section{Site}

\paragraph{}
Le site du Hackathon comprendra:

\begin{itemize}
  \item trois salles informatiques banalisées type SMIG (par exemple: 5004,
	  5006, 5008);
  \item une salle de cours banalisée (équipée de vidéo-projecteur) de plus de
	  50 places (par exemple: 4201);
  \item une salle de repos (salle foyer); et
  \item la cafétéria.
\end{itemize}

\section{Ressources informatiques à fournir par ESIEE Paris}

\begin{itemize}
	\item Les ordinateurs des salles informatiques avec un environnement de
		développement fourni conjointement par le Club*Nix et XX sous la forme
		d'une image Linux équipée des logiciels nécessaires. Cette image devra
		être déployée par le SMIG sur l'ensemble des postes des salles
		informatiques au plus tard le 1er avril à 19H. Cette image sera mise à
		disposition pour déploiement au plus tard le DATE.
	\item Une connexion informatique entre les objets connectés et les postes
		informatiques est à prévoir.
	\item Autorisation d'ouvrir un site web dynamique pour la Hackathon,
		visible depuis l'extérieur d'ESIEE Paris.
	\item Une connexion filaire pour les ordinateurs personnels des
		participants.
\end{itemize}

\section{Autres ressources demandées à ESIEE Paris}

\paragraph{}
Accès à un réfrigérateur/congélateur et à un/plusieurs four(s) à micro-onde.

\section{Ressources fournies par le Club*Nix}

\begin{itemize}
	\item Des boissons et de la nourriture pour tous les participants pendant
		la durée de l'évènement.
\end{itemize}

\section{Ressources fournies par l'entreprise DataPublica }

\begin{itemize}
	\item Les jeux de données pour tous les thèmes hors objets connectés. ??
\end{itemize}

\section{Soutiens}

\paragraph{}
L'organisation du Hackathon par le Club*Nix est déjà soutenue par:

\begin{itemize}
	\item le département informatique d'ESIEE Paris;
	\item François Bancilhon (CEO de Data Publica), le parrain de la filière
		informatique et membre du conseil scientifique d'ESIEE Paris.
\end{itemize}

\paragraph{}
Au moyen de ce document, nous souhaitons également demander le soutien de la
direction général d'ESIEE Paris, de la direction du développement. Nous
souhaitons également solliciter des partenaires industriels d'ESIEE Paris pour:

\begin{itemize}
	\item participer au jury; et
	\item remettre des prix à l'issue du jury.
\end{itemize}

\section{Références}

\paragraph{}
Site web d'aide à l'organisation d'un Hackathon: \url{https://hackathon.guide/}

\end{document}

%%  LocalWords:  ESIEE Hackathon nix d'ESIEE Bancilhon CEO Publica JM
%%  LocalWords:  Rémi Élodie Caroy l'ESIEE SMIG Linux web micro-onde
%%  LocalWords:  DataPublica Perrin Eric Rahingomanana Didier Degny
%%  LocalWords:  Krys Markowski Wirth Pointet EPITA
